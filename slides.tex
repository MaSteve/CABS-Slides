\documentclass[spanish, a4paper, 12pt, final, slideColor, nototal, colorBG, pdf, noaccumulate, darkblue] {beamer}
\usepackage[spanish]{babel}
\usepackage[utf8]{inputenc}
\usepackage{amsmath}
\usepackage{amssymb}
\usepackage{amsfonts}
\usepackage{latexsym}
\usepackage{mathtools}
\usepackage{anysize}
%\marginsize{2cm}{2cm}{2cm}{3cm}
\usepackage{soul}
\usepackage{mathtools}
\DeclarePairedDelimiter{\ceil}{\lceil}{\rceil}
\newcommand\eqdef{\stackrel{\mathclap{\mbox{\tiny{def}}}}{=}}
\newcommand\eqac{\stackrel{\mathclap{\mbox{*}}}{=}}

\usepackage{graphicx}
\usepackage{hyperref}
\usepackage{float}
\usepackage{verbatim}
\usepackage{caption}
\captionsetup{font=scriptsize,labelfont=scriptsize}
\DeclareGraphicsExtensions{.pdf,.png,.jpg}

\usetheme{Madrid}

\title{CABS}
\subtitle{A C-like language}
\author{Marco Antonio Garrido Rojo\thanks{\url{https://github.com/MaSteve/CABS-Slides}}}
\date{\today}

\begin{document}
\maketitle
\begin{frame}
  \frametitle{Introducción}
  \begin{itemize}
    \item Con el avance de los ordenadores durante las últimas décadas del siglo pasado surgió el desarrollo de programas concurrentes que aprovecharan mejor los recursos.
    \item Hoy en día es común el uso de esta práctica en la mayoría de aplicaciones y programas.
    \item La ejecución paralela conlleva riesgos adicionales como deadlocks o condiciones de carrera no presentes en los programas secuenciales.
  \end{itemize}
\end{frame}
\begin{frame}
  \frametitle{Introducción}
  \begin{itemize}
    \item Cada vez es más necesario el uso de técnicas de validación como el testing.
    \item En esta línea es necesario detectar la redundancias de las distintas trazas de ejecución de un programa para crear tests más efectivos.
  \end{itemize}
\end{frame}
\begin{frame}
  \frametitle{ABS, SYCO y aPET}
  \begin{itemize}
    \item ABS es un lenguaje de modelado, orientado a objetos y concurrente que usa un modelo de paso de mensajes entre actores.
    \item Emplea llamadas asíncronas a objetos aislados unos de otros en términos de memoria.
    \item Cada objeto es capaz de gestionar un mensaje a la vez (e.g. reducción de la explosión de estados).
    \item Es posible usarlo con herramientas especializadas como SYCO y aPET para analizar implementaciones y generar tests para ellas.
  \end{itemize}
\end{frame}
\begin{frame}
  \frametitle{ABS y sus inconvenientes}
  \begin{itemize}
    \item Sintaxis compleja: interfaces y clases que las implementan, llamadas asíncronas\dots
    \item Boilerplate
    \item Alejado de las implementaciones reales (e.g. modelado e implementación separados).
  \end{itemize}
\end{frame}
\begin{frame}
  \frametitle{¿Qué es CABS?}
  La idea principal es tener un lenguaje de programación sencillo que
  \begin{itemize}
    \item Tenga una sintaxis similar a C.
    \item Permita ejecuciones concurrentes de grano fino de forma simple (mismo comportamiento que pthread).
    \item Tenga un tipado estricto y estático.
    \item Paradigma imperativo con funciones y arrays.
    \item Que pueda usar las mismas herramientas que ABS.
  \end{itemize}
\end{frame}
\end{document}
